%!TEX root = ../Thesis.tex*

\chapter{Abstract}
\thispagestyle{empty}

%\blindtext[2]


The widespread use of Intelligent Transportation Systems (ITS) primarily stems from the imperative to mitigate human errors contributing to accidents. However, addressing the ``individual'' Automated Vehicles (AVs) alone remains insufficient, given that a number of situations  require the  ``coordination'' of the relative movements of AVs.





Within the Multi-Vehicle System (MVS) paradigm, AVs derive advantages from connectivity and road preview information. Consequently, they can sense more accurately, process more information, and can be more tightly controlled. The collaborative assessment of safety within MVS allows for the establishment of advanced collision-avoidance strategies, particularly in challenging scenarios such as intersection crossings and on-ramp mergings. Moreover, MVS technology facilitates the reduction of gaps between vehicles, enhancing road capacity and traffic flow. The MVS's shorter response time enables improved control of AV dynamics, paving the way for promising energy-centric strategies.





The main aim of the research done in this Ph.D. manuscript is to propose a safe and energy efficient decision/control architecture for MVS that navigates in dynamic and complex environment such as on-ramp merging and multi-road navigation in highway. Based on the multi-controller foundational control architecture a Cooperative Multi-Controller Architecture (C-MCA) is proposed. 










The first part of the proposed C-MCA deals with the decision-making level. This level involves a multi-behavior decision-making strategy, responsible for activating the MVS' suitable behavior based on the safety metric. Two distinct behaviors are proposed: (a) The nominal behavior, designed for executing the merging scenario while adhering to the individual goals of the MVS vehicles. It is activated when no collision risk is detected, (b) The cooperative behavior, activated by the decision-making level when both of the safety and the energy efficiency requirement are not satisfied by the nominal behavior. Its objective is to solve the merging conflict by generating a safe and energy efficient passing order of the MVS vehicles in the merging zone. 






 The second part of the proposed C-MCA focuses on the local trajectory planning level. Each behavior is assigned a dedicated controller tailored to meet its specific requirements. For instance, the cooperative behavior controller has the responsibility of translating the vehicle's passing order into feasible dynamics (i.e., trajectory, velocity, etc.). The translation task is facilitated through the proposed dynamic formation reconfiguration strategy. In essence, the approach leverages the multi-vehicle system's formation navigation capabilities to conceptualize the merging challenge as a formation reconfiguration problem. A formalization of the formation reconfiguration problem is presented, employing a flexible and generic formal approach. The proposed dynamic formation reconfiguration strategy, employs virtual dynamic targets to ensure a secure reshaping of the formation toward the desired configuration based on the selected passing order. The overall architecture's performance is assessed through co-simulation using Matlab/Simulink and SCANeR studio. 
 
 
 


 
 \paragraph*{Keywords: } Highway cooperative navigation, Multi-vehicle system, Multi-controller architecture, Multi-behavior decision-making strategy, Dynamic formation reconfiguration strategy. 
 

 
