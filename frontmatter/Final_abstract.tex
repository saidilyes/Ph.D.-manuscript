\thispagestyle{empty}

\subsubsection*{Abstract}


{\tiny The widespread use of Intelligent Transportation Systems (ITS) primarily stems from the imperative to mitigate human errors contributing to accidents. However, addressing the ``individual'' Automated Vehicles (AVs) alone remains insufficient, given that a number of situations  require the  ``coordination'' of the relative movements of AVs. Within the Multi-Vehicle System (MVS) paradigm, AVs derive advantages from connectivity and road preview information. Consequently, they can sense more accurately, process more information, and can be more tightly controlled. The collaborative assessment of safety within MVS allows for the establishment of advanced collision-avoidance strategies, particularly in challenging scenarios such as intersection crossings and on-ramp mergings. Moreover, MVS technology facilitates the reduction of gaps between vehicles, enhancing road capacity and traffic flow. The MVS's shorter response time enables improved control of AV dynamics, paving the way for promising energy-centric strategies. The main aim of the research done in this Ph.D. manuscript is to propose a safe and energy efficient decision/control architecture for MVS that navigates in dynamic and complex environment such as on-ramp merging and multi-road navigation in highway. Based on the multi-controller foundational control architecture a Cooperative Multi-Controller Architecture (C-MCA) is proposed. The first part of the proposed C-MCA deals with the decision-making level. This level involves a multi-behavior decision-making strategy, responsible for activating the MVS' suitable behavior based on the safety metric. Two distinct behaviors are proposed: (a) The nominal behavior, designed for executing the merging scenario while adhering to the individual goals of the MVS vehicles. It is activated when no collision risk is detected, (b) The cooperative behavior, activated by the decision-making level when both of the safety and the energy efficiency requirement are not satisfied by the nominal behavior. Its objective is to solve the merging conflict by generating a safe and energy efficient passing order of the MVS vehicles in the merging zone. The second part of the proposed C-MCA focuses on the local trajectory planning level. Each behavior is assigned a dedicated controller tailored to meet its specific requirements. For instance, the cooperative behavior controller has the responsibility of translating the vehicle's passing order into feasible dynamics (i.e., trajectory, velocity, etc.). The translation task is facilitated through the proposed dynamic formation reconfiguration strategy. In essence, the approach leverages the multi-vehicle system's formation navigation capabilities to conceptualize the merging challenge as a formation reconfiguration problem. A formalization of the formation reconfiguration problem is presented, employing a flexible and generic formal approach. The proposed dynamic formation reconfiguration strategy, employs virtual dynamic targets to ensure a secure reshaping of the formation toward the desired configuration based on the selected passing order. The overall architecture's performance is assessed through co-simulation using Matlab/Simulink and SCANeR studio. 

 \textbf{\scriptsize Keywords: }  Highway cooperative navigation, Multi-vehicle system, Multi-controller architecture, Multi-behavior decision-making strategy, Dynamic formation reconfiguration strategy. }



\vspace{0.3cm}



 \subsubsection*{Résumé}

 
{\tiny L'engouement important pour  les systèmes de transport intelligent est justifié principalement par l'impératif de réduire, voire annihiler, les erreurs humaines induisant les accidents. Cependant, s'attaquer uniquement aux Véhicules Autonomes (VAs) ``individuels'' demeure insuffisant, étant donné que plusieurs situations nécessitent la ``coordination'' des mouvements relatifs des VAs. Dans le paradigme du Système Multi-Véhicules (SMV), les VAs bénéficient de l'information issue de leur connectivité. Par conséquent, ils peuvent détecter plus précisément, traiter davantage d'informations et être contrôlés de manière plus précise. L'évaluation collaborative de la sécurité au sein du SMV permet l'établissement de stratégies avancées en matière d'évitement de collision, en particulier dans des scénarios complexes tels que les croisements au sein d'intersections et les insertions sur les entrées d'autoroute. De plus, la technologie SMV facilite la réduction des espaces entre les véhicules, améliorant ainsi la capacité et la fluidité du trafic routier. Le temps de réponse plus court du SMV permet un meilleur contrôle de la dynamique des VAs, ouvrant la voie à des stratégies énergétiques prometteuses. L'objectif principal des travaux de recherche constituant ce manuscrit de doctorat est de proposer une architecture de décision/contrôle sûre et peu énergivore pour le SMV naviguant dans des environnements dynamiques et complexes. Inspirée des architectures multi-contrôleurs, une Architecture Multi-Contrôleurs Coopérative est proposée. La première partie de l'architecture proposée concerne le niveau de prise de décision. Ce niveau, impliquant une stratégie de prise de décision à plusieurs comportements, est responsable de l'activation du comportement approprié du SMV en fonction de la métrique de sécurité. Deux comportements distincts sont proposés : (a) Le comportement nominal, conçu pour réaliser le scénario d'insertion tout en respectant les objectifs individuels des véhicules formant le SMV, est activé lorsqu'aucun risque de collision est détecté, (b) Le comportement coopératif est activé par le niveau de prise de décision lorsque l'exigence de sécurité n'est pas satisfaite par le comportement nominal. Son objectif est de résoudre le conflit lors de l'insertion en générant un ordre de passage sûr et économe énergétiquement pour les véhicules composant le SMV dans la zone d'insertion. La deuxième partie de l'architecture proposée se concentre sur le niveau de planification de trajectoire locale. Chaque comportement se voit attribuer un contrôleur dédié, conçu pour répondre à ses besoins spécifiques. Par exemple, le contrôleur du comportement coopératif est chargé de traduire l'ordre de passage des véhicules en dynamiques réalisables (trajectoire, vitesse, etc.). Cette tâche d'obtention de dynamiques réalisables est facilitée par la stratégie de reconfiguration dynamique de la formation proposée. En substance, l'approche tire parti des capacités de navigation en formation du SMV pour conceptualiser des manœuvres coopératives ayant trait d'insertion en milieu autoroutier, comme un problème de reconfiguration de formation. Une formalisation du problème de reconfiguration de la formation est présentée, utilisant une approche formelle, flexible et générique. La stratégie proposée utilise des cibles dynamiques virtuelles pour garantir une reconfiguration sécurisée de la formation, ceci, de sa forme initiale vers la configuration souhaitée par rapport à l'ordre de passage sélectionné. La performance de l'architecture globale a été évaluée par co-simulation en utilisant Matlab/Simulink et SCANeR studio.

\textbf{\scriptsize Mots-clés :}  Navigation coopérative en milieu autoroutier, Système multi-véhicules, Architecture multi-contrôleurs, Stratégie multi-comportementales de prise de décision, Stratégie de reconfiguration dynamique d'une formation.}